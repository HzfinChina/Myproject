\documentclass[utf8]{ctexart}
\author{机械一班\\210310119\\胡志烽}
\title{中国企业家面试题目}
\date{}
\usepackage{geometry}
\geometry{a4paper,left=1.5cm,right=1.5cm,top=1.5cm,bottom=1.5cm}
\begin{document}
\maketitle
\section{}
我会有意识地对现实生活中的现象进行批判性思考。比如在最近一则港大禁止使用 Chatgpt 的新闻中,我会意识到大学采取这种强制措施背后的本质是培养学生独立思考和学习知识和知识的验证过程与使用人工智能带来的便利的矛盾。
\section{}
商业模式可以被复制。商业模式是一种可以被复制的商业结构,它描述了一家公司如何创造价值,如何把这种价值传递给客户,以及如何获得收益。商业模式可以被复制,因为它提供了一种可以被复制的商业结构,可以被其他公司采用,以获得相同的成功。
\section{}
我认为我会是一个评估者。在团队提出方案后,我可以深入评估每个方案的优缺点,对团队的方案提出改进建议。
\section{}
学到在一家企业是如何运营管理的知识,和同学分担责任完成任务。
\section{}
\begin{center}
\heiti{知识协调、分布式认知和交易创新\\在创造企业和国家竞争优势中的作用:一个以理论为基础的中国无人机产业案例研究}
\end{center}
\heiti{摘要: }

\noindent\songti{基于交易创新的概念,我们构建了一个理论框架,作为国家竞争优势的驱动力,通过企业的企业家活动进行调节。这建立在最近的文献上,强调国际贸易网络的“超专业化”,而不是比较优势的一般作用。我们认为,交易创新涉及不同形式的社会和组织资本、技术和社会认知的复杂相互作用。后者现象在我们应用理论框架的案例研究中至关重要,即中国无人机行业,其中一家公司成为领导者。社会认知涉及通过数字媒体积累公开可访问的知识,并直接与珠江三角洲的分散和模块化生产结构相互作用。这使得在具有巨大增长潜力的高度特定的工业领域有效地利用比较优势,并结合特定技能、知识和能力的发展。}

\noindent\heiti{关键词:} 竞争优势;技能;社交认知;互联网;无人机;中国

\end{document}
