\documentclass[utf8]{ctexart}
\xeCJKsetup{CJKmath=true}
\usepackage{geometry}
\geometry{a4paper,left=1.5cm,right=1.5cm,top=1.5cm,bottom=1.5cm}
% Math Support 
\usepackage{amsmath,amsfonts,amssymb,amsthm} %数学符号
% Codingstyle 
\usepackage{listings}
\lstset{
language = C,
numbers = left,
frame = shadowbox
}

\author{胡志烽}
\linespread{1.5}
\title{第一章作业}
\begin{document}
\maketitle
\section*{2.1}
按增长率排列下列函数:
\begin{enumerate}
    \item $N$ \item $\sqrt N$ \item $N^{1.5}$ \item $N^2$ \item $N\log N$ \item $N \log N$ \item $N\log \log N$ \item $N \log ^2 N$ \item $N\log (N^2)$ \item $2/N$ 
    \item $2^N$
    \item $2^{N/2}$ 
    \item $37$
    \item $N^2 \log N$
    \item $N^3$
\end{enumerate}

\begin{proof}[解]
    \begin{gather*}
        2^ N > 2^{N/2}>N^3 > N^2 \log N > N ^2 \\
        我们来比较\sqrt N 和\log N  \\ 
        \sqrt N > \log N\ 
        所以N^{1.5}>N\log ^2 N>N\log (N^2) >N\log N >N\log \log N\\
        综上,2^ N > 2^{N/2}>N^3 > N^2 \log N > N ^2 > N^{1.5} >N\log ^2 N>N\log (N^2) >N\log N >N\log \log N\\
        >N>\sqrt N >37 >2/N
    \end{gather*}
\end{proof}

\section*{2.2} 
suppose $T_1 (N) = O(f(N)) ,T_2 (N) = O(f(N))$ which of below eqations hold
\begin{enumerate}
    \item $T_1(N) + T_2(N) = O(f(N))$
        \begin{proof}
            \begin{gather*}
                \text{by definition } \exists c_1, n_1 > 0,when\ N > n_1, T_1(N) \le c_1f(N)
                \\ \exists c_2, n_2 >0, when\ N > n_2, T_2 (N) \le c_2f(N)\\
                let\ n_0 = \max (n_1, n_2), c_0 = \max (c_1, c_2),when\ N > n_0 , T_1(N) + T_2(N) \le c_0 f(N)
            \end{gather*}
        \end{proof}
    \item $T_1(N) - T_2(N) = o(f(N))$
        不成立。反例:$T_1(N ) = 2N, T_2(N) = N, f(N) = N$
    \item $\dfrac{T_1(N)}{T_2(N)} = O(1)$ 
        不成立。反例:$f(N) = N^2 , T_1 (N) = N^2, T_2(N) = N$
    \item $T_1(N) = O(T_2 (N))$
        不成立。见3
\end{enumerate}

\section*{2.3}
哪个函数增长得更快?$N \log N, N^{1+ \varepsilon / \sqrt{\log N}}(\varepsilon > 0)$
\begin{proof}[解]
    \begin{gather*}
        \text{两边同时取对数 } \log N + \log \log N , (1 + \varepsilon /\sqrt {\log N})\log N\\ 
        \text{即}\log\log N , \varepsilon \sqrt{\log N} \\
        \varepsilon \sqrt{\log N } > \log \log N \\
        \text{因此} N \log N< N^{1+ \varepsilon / \sqrt{\log N}}(\varepsilon > 0)
    \end{gather*}
\end{proof}

\section*{2.4}
证明对任意常数$k$, $\log ^ k N = o(N)$
\begin{proof}
\begin{gather*}
    1. k \ge 1,略\\
    2. k <  0, \log ^k N 递减\\
    3. 0 \le k < 1  \lim_{N \to \infty}\frac{\log ^k N}{N} = \lim_{N \to \infty} k \frac{\log ^{k-1} N }{N\ln 2} = 0
\end{gather*}
\end{proof}

\section*{2.5}
定义两个函数$f(N),g(N)$,使得$f(N) \ne O(g(N)),g(N) \ne O(f(N))$
\begin{proof}[解]
    \begin{gather*}
    f(N) = \begin{cases}
        0& N为奇\\ 1 &N为偶
    \end{cases},
    g(N) = \begin{cases}
        1 &N 为偶 \\ 0 & N 为奇
    \end{cases}
    \end{gather*}
\end{proof}

\section*{2.6}
对6个程序片段中的每一个:
\begin{enumerate}
    \item 给出运行时间分析
    \item 编程,用实际的运行时间与分析比较。
\end{enumerate}
\end{document}
