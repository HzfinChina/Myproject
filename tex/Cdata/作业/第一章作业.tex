\documentclass[utf8]{ctexart}
\usepackage{geometry}
\geometry{a4paper,left=1.5cm,right=1.5cm,top=1.5cm,bottom=1.5cm}
% Math Support 
\usepackage{amsmath,amsfonts,amssymb,amsthm} %数学符号
% Codingstyle 
\usepackage{listings}
\lstset{
language = C,
numbers = left,
frame = shadowbox
}

\author{胡志烽}
\linespread{1.5}
\title{第一章作业}
\begin{document}
\maketitle

\section*{1.5}
\begin{enumerate}
\item 证明$\log X <X$ 对所有$X>0$成立
\begin{proof}
\[
X - \log X > X- \log(X+1)>X-\ln(X+1)
\]
\end{proof}
\item 证明$\log (A^B) = B\log A$
\begin{proof}
\[
suppose\ \begin{cases}
x = \log (A^B)\\ y = B\log A
\end{cases}
\ then\ \begin{cases}
2^x = A^B\\ 2^{y/B} = A
\end{cases}
\therefore 2^y = A^B 
\]
\end{proof}
\end{enumerate}

\section*{1.6}
\begin{enumerate}
\item $\displaystyle\sum_{i=0}^{\infty} \frac{1}{4^i}$ 
\begin{proof}[解]
等比数列求和
\[ 
Sum = a_0 \times \frac{1-q^n}{1-q} = 1 \times \lim_{n \to \infty}\frac{1-(\frac 14)^n}{\frac 34} = \frac 43
\]
\end{proof}
\item $\displaystyle \sum_{i =0 }^{\infty} \frac i{4^i}$
\begin{proof}[解]
\begin{align*}
suppose\ S &= \sum_{i = 0}^{N}  \frac i {4^i} \\ 
then\ \frac 14 S  &= \sum_{i = 0}^{N} \frac{i}{4^{(i+1)}}\\
\therefore\ \frac 34 S  &=  (\frac 14 + \frac 2{4^2} + \frac 3{4^3} + \cdots + \frac N {4^N}) - (\frac 1 {4^2} + \frac 2{4^3} +\cdots + \frac N{4^{N+1}})\\
& = \frac 14 +\frac 1 {4^2} +\cdots \frac 1{4^N} - \frac N{4^{N+1}}\\
& = \frac 14\times  \frac{1-(\frac 14)^N}{1-\frac 14} - \frac{N}{4^{N+1}}\\ 
S& =  \frac 49\left[1-(\frac 14)^N\right] - \frac{N}{4^{N+1}}\\
\therefore \lim_{N \to \infty} S &= \frac 49
\end{align*}
\end{proof}
\item $\displaystyle \sum_{i=0}^{\infty} \frac{i^2}{4^i}$
\begin{proof}[解]
\begin{align*}
S &= \frac 14 + \frac{2^2}{4^2} + \frac{3^2}{4^3} +\cdots\\
4S &= 1+ \frac 44 + \frac {3^2} {4^2} +\cdots \\
\therefore 3S & = 1 + \frac 34 + \frac 5{4^2} +\cdots + \frac{2n+1}{4^n} + \cdots\\
& =  2 \sum_{i=0}^{\infty} \frac i {4^i} + \sum_{i = 0}^{\infty} \frac 1{4^i}\\
\therefore S & = 2\times \frac 49 + \frac 43 = \frac {20}9
\end{align*} 
\end{proof}
\item $\displaystyle \sum_{i=0}^{\infty} \frac{i^N}{4^i}$:略
\end{enumerate}
\section*{1.7}
估计
\[\sum_{i=\lfloor N/2\rfloor}^{N}\frac 1i\]
\begin{proof}[解]
\begin{align*}
\text{由调和和}H_N = \sum_{i=1}^{N} \frac 1i \approx \log _e N\\ 
\sum_{i=\lfloor N/2\rfloor}^{N}\frac 1i = \sum_{i=1}^{N} \frac 1i - \sum_{i=1}^{\lfloor N/2\rfloor -1} \frac 1i\approx \ln N - \ln \frac N 2 = \ln 2
\end{align*}
\end{proof}
\section*{1.8}
\[2^{100}(\mod 5)\]

\begin{proof}[解]
\begin{align*}
2^n \mod 5 & = k\\
2^{n+1}\mod 5 & = 2k\mod 5\\
let k &= 1\\
&\downarrow\\
2^n\mod 5 =1, 2^{n+1}\mod = 2&,2^{n+2}\mod = 4,2^{n+3}\mod=3,2^{n+4}\mod =1\\
\therefore 2^N\mod 5\ T &= 4\\
\therefore 2^100 \mod 5 = 2^0\mod 5 =1
\end{align*}
\end{proof}
\section*{1.9}
let $F_i = $ Fibonacci sequence($F_0  = F$),Proof:
\begin{enumerate}
\item $\displaystyle \sum_{i=1}^{N-2} F_i = F_N -2$
\begin{proof}
\begin{align*}
    \text{Obviously when }N-2&=1 &\sum_{i=1}^{1}  = F_3 -2 \\
\text{suppose that when }N-2 &= k &\sum_{i= 1}^{k} F_i = F_{k+2}-2\\
\text{when }N-2 & = k+1 &\sum_{i=1}^{k+1} F_i = \sum_{i=1}^{k} F_i + F_{k+1} = F_{k+2} -2  + F_{k+1} = F_{k+3} -2\\
\end{align*}
\end{proof}
\item $F_N < \phi ^N,\phi  = (1+\sqrt 5 )/2$
    \begin{proof}
    \begin{align*}
    \phi ^2 = \frac 32 + \frac{\sqrt 5 }{2} = \phi  + 1 \\ 
    \therefore \phi^{-2}+\phi^{-1}  = 1\\
    \text{when } N = 1,2, F_N < \phi ^N \text{成立} \\
    \text{suppose that when }N= k, F_k <\phi ^N\\
    \text{when }N = k+1, F_{k+1} = F_k + F_{k-1} < \phi ^k + \phi ^{k-1}  = \phi^{k+1} (\phi^{-1} + \phi^{-2}) = \phi^{k+1} 
    \end{align*}
    \end{proof}
\end{enumerate}

\end{document}

